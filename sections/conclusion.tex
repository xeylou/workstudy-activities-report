\section{Conclusion et perspectives}
\label{conclusion}

Lors de cette année d'apprentissage, j'ai été amené à aborder des sujets d'études toujours plus complexes pour mes projets, pour la plupart en totale autonomie. Cette année m'a permis d'apprendre des compétences clés en entreprise que je n'aurai soupçonnées par des besoins personnels : comment monter efficacement une étude comparative et budgétaire, évoluer dans sa communication client et interne, découvrir le véritable besoin d'un client devant une demande etc.
\\ \\
%tout cela en supplément de ma montée en compétences sur tous les domaines techniques que j'ai abordés.
\\ \\
J'ai toujours essayé d'apprendre et d'évoluer dans mon travail plutôt que d'avancer le plus vite possible afin de rendre un projet rapidement. Pour ce faire, j'ai appris à documenter mon travail pour au mieux le conserver. J'en ai donc fait un suivi, mais aussi participé à des formations que je n'aurai pas besoin de faire deux fois. Pareil pour mes apprentissages d'outils : ayant écrit des procédures pour les actions les plus utiles et/ou complexes pour ces outils.
\\ \\
Je retiens que l'accomplissions d'un projet ne se termine pas lorsque nous avons fini ce qu'il nous était demandé de faire, mais quand le besoin initial relevant de celui-ci a été répondu. Ainsi, je me suis donné libre court de proposer de nouvelles directions dans mes projets, chose que je trouvais normale lorsque j'avais finalement cerné le besoin initial de ce projet et que je trouvais que la direction de son développement n'allait pas dans ce sens.